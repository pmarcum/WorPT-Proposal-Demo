%%================= BEGIN INSTRUCTIONS =================
%%   H O W   T O   U S E   T H I S   F I L E
%% (1) download or connect to https://github.com/pmarcum/LaTex-Formatting and select the
%%        set of files (aas-style or general) that meet your situation, you can either download the files or link
%%        Overleaf to the needed files in the GitHub repository through URL, and then add the following to
%%        the preamble of your main paper
%\usepackage{FUpackages}
%\usepackage{FUformatting}
%\usepackage{FUnewCommands} % not needed by WorPT files but generally useful
%%
%% (2) copy/paste the following into the main latex file where you want to put this section of text (it is not a table), and then uncomment the below lines by removing the first percent sign at the beginning of the line
%%   IMPORTANT SHORTCUT:  You can toggle commenting / un-commenting across multiple lines within Overleaf by mouse-selecting the desired block of lines to un-comment (or comment), and then hitting: control slash ( [Ctrl] / )   ... or ...  command slash  ([command] /)  for a PC or Mac, resp.
%%        (Note that lines that should remain commented-out in the below block have 2 percent signs ... the first will be removed when you uncomment the whole block, but the second percent sign will remain as intended to retain that line as a true comment within the Latex document.)
%% > > > > > > > > > > > > > < < < < < < < < < < < < < <
%% Block of code to be uncommented by removing the *first* percent-sign at the beginning of the following lines below (leave intentional comments with at least 1 percent sign to keep them commented!)
%% The next couple of lines define cell and font colors that can be changed if desired (keep this comment line commented-out!)
%\def\myNameBoldface#1{\textbf{#1}} % Makes the name and role at the beginning of each paragraph bold-faced; change {\textbf{#1}} to {#1} or to other preference.
%%
%\clearpage
%%======================================================
%     H O W   T O   U S E   T H I S   F I L E
% (1)  add the following in the preamble of your main paper:
%          \makeatletter \def\expinput#1{\@@input#1} \makeatother
% (2) type the following into the main latex file where you want to put this section of text (it is not a table):
%\expinput{do_NOT_manually_edit/NOTANONteamSummaries}
%======================================================
Dr. \textbf{Dan J. Dent{\'{o}}n, III}, \textbf{PI}, will lead the emission maps production, and the reduction of image mosaics related to archive applications. Finally, he will lead the first publication, a paper describing the pipeline and improved images.  His $\sim$15 years of experience in image analysis are needed for successful and timely completion of these tasks. In addition to leading these tasks, he will assist with model preparation by generating simualted data, and will identify fields of interest for the archive-related work, as well as help with code documentation and the development of the second publication. \\ \\
Dr. \textbf{Sally K. Smith}, \textbf{Sci-PI}, will lead simulated data generation for models input and the incorporation of thermal models, as well as code documentation.  Her extensive work with model simulations and archival processes are well-matched to these roles.  Additionally, her expertise will be used to assist with generation of emission maps, identifying fields of interest, reduction of image mosaics and in the development of both publications. \\ \\
Dr. \textbf{Pamela M. Marcum}, \textbf{co-I(1)}, will lead the identification of fields and the second publication, a paper on galaxy identification.  Her background in extragalactic astronomy is essential for successful implementation of these tasks. In addition to these responsibilities, she will provide expertise in the model preparation by generating simulated input data, and in the incorporation of thermal emission models.  She will also assist with documentation and will co-author the first publication. \\ 
%%================ E N D   I N S T R U C T I O N S =============== (leave this line commented out!)
Dr. \myNameBoldface{Dan J. Dent{\'{o}}n, III}, \myNameBoldface{PI}, will lead the emission maps production, and the reduction of image mosaics related to archive applications. Finally, he will lead the first publication, a paper describing the pipeline and improved images.  His $\sim$15 years of experience in image analysis are needed for successful and timely completion of these tasks. In addition to leading these tasks, he will assist with model preparation by generating simualted data, and will identify fields of interest for the archive-related work, as well as help with code documentation and the development of the second publication. \par
Dr. \myNameBoldface{Sally K. Smith}, \myNameBoldface{Sci-PI}, will lead simulated data generation for models input and the incorporation of thermal models, as well as code documentation.  Her extensive work with model simulations and archival processes are well-matched to these roles.  Additionally, her expertise will be used to assist with generation of emission maps, identifying fields of interest, reduction of image mosaics and in the development of both publications. \par
Prof. \myNameBoldface{Gisella Z. Gala}, \myNameBoldface{coll(1)}, will assist with fields and galaxy identification and image mosaic reduction, as well as with the development of Papers~1 and 2. Her extensive experience in image analysis will significantly reduce the risk of false positive detections created by artifacts. \par
Dr. \myNameBoldface{Pamela M. Marcum}, \myNameBoldface{co-I(1)}, will lead the identification of fields and the second publication, a paper on galaxy identification.  Her background in extragalactic astronomy is essential for successful implementation of these tasks. In addition to these responsibilities, she will provide expertise in the model preparation by generating simulated input data, and in the incorporation of thermal emission models.  She will also assist with documentation and will co-author the first publication. \par