%======================================================
%     H O W   T O   U S E   T H I S   F I L E
% (1) download or connect to https://github.com/pmarcum/LaTex-Formatting and select the
%        set of files (aas-style or general) that meet your situation, you can either download the files or link
%        Overleaf to the needed files in the GitHub repository through URL, and then add the following to
%        the preamble of your main paper
%\usepackage{FUpackages}
%\usepackage{FUformatting}
%\usepackage{FUnewCommands} % not needed by WorPT files but generally useful
% (2) type the following into the main latex file where you want to put this table:
%\addtocounter{table}{-1} % corrects double-counting of longtable and table combination
%\begin{table*}[h!] %table is needed so that table caption at bottom has same width as table
%   \renewcommand{\arraystretch}{0.7} %adjust to control vertical row separation/spacing
%   \setlength{\tabcolsep}{5pt} %adjust to control horizontal column separation/spacing
%   \begin{longtable}{|p{3.3in}||c!{\color{lightgray}\vrule}c!{\color{lightgray}\vrule}!{\color{lightgray}\vrule}c!{\color{lightgray}\vrule}c||p{0.45in}!{\color{lightgray}\vrule}p{1.2in}|}
%      \expinput{do_NOT_manually_edit/isANONtasks}
%   \end{longtable}
%   \caption{\label{tab:isANONtasks} \textbf{Task Timeline:} Team member roles, rightmost column, are cross-referenced with corresponding names in the non-anonymized personnel and work effort table.  {\color{red} \textbf{Paper~1:} Sample and methods for enhancing detectability of  low SB X-ray emission, presentation of emission maps, description of database and pipeline software (which will be released in a public repository at the time of paper submission). \textbf{Paper~2:} Methodologies for measuring the gas halo size and other gas properties, analyisis of the diffuse hot gas halos as functions of galaxy properties (environment, galaxy morphology, stellar mass, and SFR based on \Chandra, \Hubble, and \Spitzer\ observations, and the SED models from the GSWLC; application of multivariate mthods to ``baseline'' the gas halo sizes (Sect.\,\ref{Sec:Baseline}). \textbf{Note~1:} See Sec.\,\ref{Sec:Sample}.}}
%\end{table*}
%======================================================
\hline
\multirow{2}{*}{\textbf{Task Description}} & \multicolumn{2}{c}{\textbf{Start}} & \multicolumn{2}{!{\color{lightgray}\vrule}c||}{\textbf{Finish}} & \textbf{Task} & \textbf{Team}\\
\cline{2-5}
\rule{0pt}{12pt} &
{\color{Red}Y} & Q & {\color{Red}Y} & Q& \textbf{Lead} &\textbf{Expertise}\\
\multicolumn{7}{|c|}{\cellcolor{Blue}\color{White}\textbf{Task A: Data and models preparation}}\\
A1: {Generate simulated data}  & {\color{Red}1} & 1 & {\color{Red}1} & 2 & Sci-PI & PI, co-I(1)\\
A2: {Make emission maps}  & {\color{Red}1} & 1 & {\color{Red}1} & 3 & PI & Sci-PI\\
A3: {Incorporate thermal emission models}  & {\color{Red}1} & 3 & {\color{Red}1} & 3 & Sci-PI & co-I(1)\\
\multicolumn{7}{|c|}{\cellcolor{Blue}\color{White}\textbf{Task B: Application to archive}}\\
B1: {Determine fields of interest}  & {\color{Red}1} & 3 & {\color{Red}2} & 1 & co-I(1) & Sci-PI, PI\\
B2: {Reduction of image mosaics}  & {\color{Red}2} & 1 & {\color{Red}2} & 3 & PI & Sci-PI\\
\multicolumn{7}{|c|}{\cellcolor{Blue}\color{White}\textbf{Task C: Documentation and publications}}\\
C1: {Code documentation \& system verification}  & {\color{Red}2} & 2 & {\color{Red}3} & 1 & Sci-PI & co-I(1), PI\\
C2: {Pub 1: pipeline and improved images}  & {\color{Red}2} & 4 & {\color{Red}3} & 2 & PI & Sci-PI, co-I(1)\\
C3: {Pub 2: galaxy identification}  & {\color{Red}3} & 1 & {\color{Red}3} & 4 & co-I(1) & Sci-PI, PI\\
\hline