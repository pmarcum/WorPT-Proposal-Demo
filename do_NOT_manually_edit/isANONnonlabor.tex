%======================================================
%     H O W   T O   U S E   T H I S   F I L E
% (1) type the following into the main latex file where you want to put this table:
%Table\,\ref{tab:isANONnonlabor} shows rolled-up values for travel costs and other non-labor costs. Travel cost details, including costing assumptions used, are presented in Table\,\ref{tab:isANONtravel}.
%\begin{table*}[h!]
%   \renewcommand{\arraystretch}{1.0} %adjust to control vertical row separation/spacing, comment-out if not needed
%   \setlength{\tabcolsep}{5pt} %adjust to control horizontal column separation/spacing, comment-out if not needed
#REF!
%      \expinput{do_NOT_manually_edit/isANONnonlabor}
%   \end{tabular}
%   \caption{\normalsize
%      \newline \newline
%      \textbf{Notes and assumptions}:
%      \newline \newline
%      {\color{red} \underline{\scshape{Equipment Costs}}: In Yrs~1-2, we request a total of \$11K for the purchase of a laptop and associated IT equipment to replace the PI's aging laptop (purchased in 2018, well past nominal 4-year refresh cycle at the time of the proposed budget period), ``NASA-tized'' computer equipment (laptops, monitors) for use by the summer interns, and as an ``emergency'' fund, should the Science~PI's $\sim$3-yr old laptop fail or need repair.}\newline \newline
%      \underline{\scshape{Travel}}: refer to Table\,\ref{tab:isANONtravel}. \newline \newline
%      {\color{red} \underline{\scshape{Publication Costs}}: Our work plan includes the publication of four key manuscripts: (see Table \ref{tab:NOTANONschedule} for details), but given the student projects, we have budgeted for 8 papers. We request a total of \$2K for publication costs,  using the assumption that the papers will fall between "Tier 1" and "Tier 2" categories as defined in ApJ/AJ guidelines. These fees are included in proposed budget.}\newline \newline
%      {\color{red}\underline{\scshape{ Materials and Supplies}}: We request an annually-averaged budget of \$1,125 to cover purchase of disk space to back up our data products and miscellaneous office and IT supplies at PI and Science-PI home institution. The distribution of these funds is top-heavy at the beginning of the grant period, when such supplies will be needed most. The disk size of the data products will be approximately ~39~Gb per exposure: four \texttt{float32} extensions per CCD, corresponding to the \textbf{1)} Zodiacal-CIB background, \textbf{2)} in-field, and \textbf{3)} out-field stray-light, and \textbf{4)} thermal emission layers, plus one \texttt{binary} extension for the Solar System object trails (\emph{streak / no streak}), for a total of 18~4096$\times$4096 detector focal plane. Storage of these products will not be required, as they will be immediately produced by the pipeline on a exposure-by-exposure basis. End-to-end simulations shall not exceed 100~Gb, and they will be accessible to the community through a public internet server. Publication~IV will require the analysis of an area equivalent to 32 adjacent field of views, the equivalent of a sector of the \RST/WFI High Latitude Wide Area Survey, which corresponds to 128 exposures. Assuming an average exposure size of 9~Gb, plus $\sim$40~Gb for the background products, we project that we will require $\sim$10--15~Tb of disk space (including backups).}}
%      \label{tab:isANONnonlabor}
%\end{table*}
%======================================================
\hline
#REF!
#REF!
\hline