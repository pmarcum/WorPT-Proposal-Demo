\documentclass[]{article}
\usepackage{CLASS_FILES/FUpackages}  % applying Frequently-Used packages
\usepackage{CLASS_FILES/FUformatting}  % applying Frequently-Used packages

%        the preamble of your main paper
%\usepackage{FUnewCommands} % not needed by WorPT files but generally useful

%      S T A R T  D O C U M E N T 
\begin{document}
How to use {\texttt{WorPT}}

\newpage
%======================================================
%     H O W   T O   U S E   T H I S   F I L E
% (1) download or connect to https://github.com/pmarcum/LaTex-Formatting and select the
%        set of files (aas-style or general) that meet your situation, you can either download the files or link
%        Overleaf to the needed files in the GitHub repository through URL, and then add the following to
%        the preamble of your main paper
%\usepackage{FUpackages}
%\usepackage{FUformatting}
%\usepackage{FUnewCommands} % not needed by WorPT files but generally useful
% (2) type the following into the main latex file where you want to put this table:
{
   \renewcommand{\arraystretch}{1.0} %adjust to control vertical row separation/spacing, comment-out if not needed
   \setlength{\tabcolsep}{5pt} %adjust to control horizontal column separation/spacing, comment-out if not needed
   \begin{longtable}{|l|*{4}{c|}}
      \expinput{do_NOT_manually_edit/isANONfte}
      \caption{\label{tab:isANONfte} Details of work efforts per member to be funded for the present proposal; {\color{red}Detailed responsibilities, tied to tasks and science goals, are provided in Sec.\,\ref{Subsec:tmeline}.}}
   \end{longtable}
 }
%======================================================

\newpage
%======================================================
%     H O W   T O   U S E   T H I S   F I L E
% (1) type the following into the main latex file where you want to put this table:
Table\,\ref{tab:isANONnonlabor} shows rolled-up values for travel costs and other non-labor costs. Travel cost details, including costing assumptions used, are presented in Table\,\ref{tab:isANONtravel}.
\begin{table*}[t!]
   \renewcommand{\arraystretch}{1.0} %adjust to control vertical row separation/spacing, comment-out if not needed
   \setlength{\tabcolsep}{5pt} %adjust to control horizontal column separation/spacing, comment-out if not needed
   \begin{tabular}{|l|*{4}{c|}}
      \expinput{do_NOT_manually_edit/isANONnonlabor}
   \end{tabular}
   \caption{\normalsize
      \newline \newline
      \textbf{Notes and assumptions}:
      \newline \newline
      {\color{red} \underline{\scshape{Equipment Costs}}: In Yrs~1-2, we request a total of \$11K for the purchase of a laptop and associated IT equipment to replace the PI's aging laptop (purchased in 2018, well past nominal 4-year refresh cycle at the time of the proposed budget period), ``NASA-tized'' computer equipment (laptops, monitors) for use by the summer interns, and as an ``emergency'' fund, should the Science~PI's $\sim$3-yr old laptop fail or need repair.}\newline \newline
      \underline{\scshape{Travel}}: refer to Table\,\ref{tab:isANONtravel}. \newline \newline
      {\color{red} \underline{\scshape{Publication Costs}}: Our work plan includes the publication of four key manuscripts: (see Table \ref{tab:NOTANONschedule} for details), but given the student projects, we have budgeted for 8 papers. We request a total of \$2K for publication costs,  using the assumption that the papers will fall between "Tier 1" and "Tier 2" categories as defined in ApJ/AJ guidelines. These fees are included in proposed budget.}\newline \newline
      {\color{red}\underline{\scshape{ Materials and Supplies}}: We request an annually-averaged budget of \$1,125 to cover purchase of disk space to back up our data products and miscellaneous office and IT supplies at PI and Science-PI home institution. The distribution of these funds is top-heavy at the beginning of the grant period, when such supplies will be needed most. The disk size of the data products will be approximately ~39~Gb per exposure: four \texttt{float32} extensions per CCD, corresponding to the \textbf{1)} Zodiacal-CIB background, \textbf{2)} in-field, and \textbf{3)} out-field stray-light, and \textbf{4)} thermal emission layers, plus one \texttt{binary} extension for the Solar System object trails (\emph{streak / no streak}), for a total of 18~4096$\times$4096 detector focal plane. Storage of these products will not be required, as they will be immediately produced by the pipeline on a exposure-by-exposure basis. End-to-end simulations shall not exceed 100~Gb, and they will be accessible to the community through a public internet server. Publication~IV will require the analysis of an area equivalent to 32 adjacent field of views, the equivalent of a sector of the \RST/WFI High Latitude Wide Area Survey, which corresponds to 128 exposures. Assuming an average exposure size of 9~Gb, plus $\sim$40~Gb for the background products, we project that we will require $\sim$10--15~Tb of disk space (including backups).}}
      \label{tab:isANONnonlabor}
\end{table*}
%======================================================

\newpage
%======================================================
%     H O W   T O   U S E   T H I S   F I L E
% (1) download or connect to https://github.com/pmarcum/LaTex-Formatting and select the
%        set of files (aas-style or general) that meet your situation, you can either download the files or link
%        Overleaf to the needed files in the GitHub repository through URL, and then add the following to
%        the preamble of your main paper
%\usepackage{FUpackages}
%\usepackage{FUformatting}
%\usepackage{FUnewCommands} % not needed by WorPT files but generally useful
% (2) type the following into the main latex file where you want to put this table:
\addtocounter{table}{-1} % corrects double-counting of longtable and table combination
\begin{table*}[t!] %table is needed so that table caption at bottom has same width as table
   \renewcommand{\arraystretch}{0.7} %adjust to control vertical row separation/spacing
   \setlength{\tabcolsep}{5pt} %adjust to control horizontal column separation/spacing
   \begin{longtable}{|p{3.3in}||c!{\color{lightgray}\vrule}c!{\color{lightgray}\vrule}!{\color{lightgray}\vrule}c!{\color{lightgray}\vrule}c||p{0.45in}!{\color{lightgray}\vrule}p{1.2in}|}
      \expinput{do_NOT_manually_edit/isANONtasks}
   \end{longtable}
   \caption{\label{tab:isANONtasks} \textbf{Task Timeline:} Team member roles, rightmost column, are cross-referenced with corresponding names in the non-anonymized personnel and work effort table.  {\color{red} \textbf{Paper~1:} Sample and methods for enhancing detectability of  low SB X-ray emission, presentation of emission maps, description of database and pipeline software (which will be released in a public repository at the time of paper submission). \textbf{Paper~2:} Methodologies for measuring the gas halo size and other gas properties, analyisis of the diffuse hot gas halos as functions of galaxy properties (environment, galaxy morphology, stellar mass, and SFR based on \Chandra, \Hubble, and \Spitzer\ observations, and the SED models from the GSWLC; application of multivariate mthods to ``baseline'' the gas halo sizes (Sect.\,\ref{Sec:Baseline}). \textbf{Note~1:} See Sec.\,\ref{Sec:Sample}.}}
\end{table*}
%======================================================

\newpage
%======================================================
%     H O W   T O   U S E   T H I S   F I L E
% (1) type the following into the main latex file where you want to put this table:
The below table provides travel costs:
\begin{table*}[t!]
   \renewcommand{\arraystretch}{1.0} %adjust to control vertical row separation/spacing, comment-out if not needed
   \setlength{\tabcolsep}{5pt} %adjust to control horizontal column separation/spacing, comment-out if not needed
   \begin{tabular}{|lcl >{\columncolor[gray]{0.85}[\tabcolsep][7pt]}c>{\columncolor[gray]{0.85}[\tabcolsep][7pt]}c>{\columncolor[gray]{0.85}[\tabcolsep][7pt]}l>{\columncolor[gray]{0.85}[\tabcolsep][7pt]}l>{\columncolor[gray]{0.85}[\tabcolsep][7pt]}ll|}
       \expinput{do_NOT_manually_edit/isANONtravel}
    \end{tabular}
   \caption{\normalsize
      \newline \newline
      \textbf{Notes and assumptions}:
      \newline \newline
      While final destinations are not known at this time, domestic and international costs are estimated based on values taken from NASA Travel Guidebook using historical averages for a 5-- and 5--day conference for U.S. and European cities, resp.,  likely to host topical meetings aligned with the science of the proposed work. Domestic lodging and per diem rates are set by the GSA; international lodging and per diem are set by the Dept. of State (note that M\&IE is included in the per diem values shown here).
      \newline \newline\noindent {\color{red} Yrs~1-2 funds will be used to present pre-publication findings at science conferences and potentially to fund trips for collaboration with team members (i.e., NASA/GSFC).}
      \newline \newline\underline{\scshape{domestic}}: per diem$+$M\&IE, car rental/day   at \$264 and \$60, resp.
      \newline \newline\underline{\scshape{international}}: per diem$+$M\&IE, public transport/day estimated at \$320 and \$100, resp.
      \newline \newline \underline{\scshape{Travel Per Team Member}} (summed over 3-year grant):\newline
      \smallIndent \textbf{PI:} 5 domestic trips; 2 intern'l trips;\newline
      \smallIndent \textbf{co-I(1):} 4 domestic trips; 1 intern'l trips;\newline
      \smallIndent \textbf{Sci-PI:} 6 domestic trips; 3 intern'l trips;\newline
      \newline
      All travel will be to present science results of this project at conferences and/or visits to home institutions of the team members for in-person collaboration. Note that above values above do not include institutional overhead.}
   \label{tab:isANONtravel}
\end{table*}
%======================================================

\clearpage
%======================================================
%     H O W   T O   U S E   T H I S   F I L E
% (1) download or connect to https://github.com/pmarcum/LaTex-Formatting and select the
%        set of files (aas-style or general) that meet your situation, you can either download the files or link
%        Overleaf to the needed files in the GitHub repository through URL, and then add the following to
%        the preamble of your main paper
%\usepackage{FUpackages}
%\usepackage{FUformatting}
%\usepackage{FUnewCommands} % not needed by WorPT files but generally useful
% (2) type the following into the main latex file where you want to put this section of text (it is not a table):
%======================================================
%     H O W   T O   U S E   T H I S   F I L E
% (1) download or connect to https://github.com/pmarcum/LaTex-Formatting and select the
%        set of files (aas-style or general) that meet your situation, you can either download the files or link
%        Overleaf to the needed files in the GitHub repository through URL, and then add the following to
%        the preamble of your main paper
%\usepackage{FUpackages}
%\usepackage{FUformatting}
%\usepackage{FUnewCommands} % not needed by WorPT files but generally useful
% (2) type the following into the main latex file where you want to put this section of text (it is not a table):
%\clearpage  % or you might want to use \newpage or \pagebreak, whatever works best
%%======================================================
%     H O W   T O   U S E   T H I S   F I L E
% (1) download or connect to https://github.com/pmarcum/LaTex-Formatting and select the
%        set of files (aas-style or general) that meet your situation, you can either download the files or link
%        Overleaf to the needed files in the GitHub repository through URL, and then add the following to
%        the preamble of your main paper
%\usepackage{FUpackages}
%\usepackage{FUformatting}
%\usepackage{FUnewCommands} % not needed by WorPT files but generally useful
% (2) type the following into the main latex file where you want to put this section of text (it is not a table):
%\clearpage  % or you might want to use \newpage or \pagebreak, whatever works best
%%======================================================
%     H O W   T O   U S E   T H I S   F I L E
% (1) download or connect to https://github.com/pmarcum/LaTex-Formatting and select the
%        set of files (aas-style or general) that meet your situation, you can either download the files or link
%        Overleaf to the needed files in the GitHub repository through URL, and then add the following to
%        the preamble of your main paper
%\usepackage{FUpackages}
%\usepackage{FUformatting}
%\usepackage{FUnewCommands} % not needed by WorPT files but generally useful
% (2) type the following into the main latex file where you want to put this section of text (it is not a table):
%\clearpage  % or you might want to use \newpage or \pagebreak, whatever works best
%\input{do_NOT_manually_edit/NOTANONbiosketches}
%================ E N D   I N S T R U C T I O N S ===============
\textbf{\color{Blue}\large Dr. Daniel Dent{\'o}n (Principal Investigator):}\\
University of CBA\\
\vspace{4ex}
\textbf{Education}\\
01012004 - 12312008, University of ABC, Ph.D., astrophysics\\
01012000 - 12312003, University of XYZ, B.S., astronomy\\
\textbf{Appointments}\\
01012012 - present, professor, University of CBA\\
01032009 - 01082011, postdoc, University of ZYX\\
\textbf{Additional Awards, Positions, Fellowships and Proposals}\\
01102011 - 01102012, Awarded 100 hrs on Gigantic Telescope\\
01052011, Best Researcher of the Year\\
\textbf{Publications relevant for the proposal:}\\
{\scriptsize{$\bullet$}} Young, E.T., Herter, T.L., Gusten, R., et al. (inc. \textbf{Denton, D.}), 2021, "Early science results from SOFIA", 8444, 844410\\
{\scriptsize{$\bullet$}} Temi, P., \textbf{Denton, D.}, Miller, W.E., et al., 2018, "SOFIA observatory performance and characterization", SPIE, 84444, 8444414\\
{\scriptsize{$\bullet$}} Gehrz, R.D., Becklin, E.E., et al. (inc.\textbf{Denton, D.}), 2015, "Status of the Stratospheric Observatory for Infrared Astronomy (SOFIA)", Adv. in Space Research, 48, 1004\\
{\scriptsize{$\bullet$}} \textbf{Denton, D.}, Appleton, P.N., Higdon, J.L, 2010, "Large Infrared and Optical Color Gradients in the Carwheel Ring Galaxy: Evidence fo rthe First Epoch of Star Formation in the Wake of an Expanding Ring", ApJ, 399, 57\\
\medskip \hrule \vspace{5pt} \medskip
\textbf{\color{Blue}\large Dr. Sally Smith (Science PI):}\\
University of CBA\\
\vspace{4ex}
\textbf{Education}\\
01012014 - 12312020, University of lmn, Ph.D., astronomy\\
01012010 - 12312014, University of abc, B.S., physics\\
\textbf{Appointments}\\
01012023 - present, research fellow, University of CBA\\
01032020 - 01082022, postdoc, University of ZYX\\
\textbf{Additional Awards, Positions, Fellowships and Proposals}\\
01102011 - 01102012, Awarded 100 hrs on Gigantic Telescope\\
43332, Teaching assistant award\\
\textbf{Publications relevant for the proposal:}\\
{\scriptsize{$\bullet$}} Lopez-Rodriguez, E., Mao, S.A., Beck, R., et al. (inc. \textbf{Smith, S.}), 2022, "Extragalactic magnetism with SOFIA (SALSA Legacy Program) IV: Program overview and first results on the polarization fraction" (submitted to ApJ)\\
{\scriptsize{$\bullet$}} Young, E.T., Herter, T.L., Gusten, R., et al. (inc. \textbf{Smith, S.}), 2021, "Early science results from SOFIA", 8444, 844410\\
{\scriptsize{$\bullet$}} Gehrz, R.D., Becklin, E.E., et al. (inc.\textbf{Smith, S.}), 2015, "Status of the Stratospheric Observatory for Infrared Astronomy (SOFIA)", Adv. in Space Research, 48, 1004\\
{\scriptsize{$\bullet$}} \textbf{Smith, S.}, Appleton, P.N., Higdon, J.L, 2010, "Large Infrared and Optical Color Gradients in the Carwheel Ring Galaxy: Evidence fo rthe First Epoch of Star Formation in the Wake of an Expanding Ring", ApJ, 399, 57\\
\medskip \hrule \vspace{5pt} \medskip
\textbf{\color{Blue}\large Dr. Pamela M. Marcum (co-Investigator):}\\
NASA ARC\\
\vspace{4ex}
\textbf{Education}\\
1994, University of Wisconsin, Ph.D., Astrophysics\\
1989, Florida Institute of Technology, M.S., Space Science\\
1989, Florida Institute of Technology, M.S., Physics\\
1987, Florida Institute of Technology, B.S., Space Science\\
\textbf{Appointments}\\
2016 - present, research scientist, NASA Ames Research Center\\
2020 - 2021, program officer (on detail), NASA HQ, Washington, D.C.\\
2009 - 2016, SOFIA project scientist, NASA Ames Research Center\\
2005 - 2008, program officer (IPA), NASA HQ, Washington, D.C.\\
2003 - 2009, associate professor, Texas Christian University (Fort Worth, TX)\\
1996 - 2002, assistant professor, Texas Christian University (Fort Worth, TX)\\
1994 - 1996, postdoctoral fellow, University of Virginia\\
\textbf{Additional Awards, Positions, Fellowships and Proposals}\\
2018 - present, Roman Space Telescope Standing Review Board\\
2021 - 2022, SOFIA Cycle 9 observing proposal 09\_0113: "Testing for multi-component magnetic fields and their effects on the structure of galaxtic disks", 20.1h, SOFIA/HAWC+ (pending observations/funding)\\
2021, VLA/21A-043,"Flares, breaks and warps in the outskirts of the HI and stellar disk of UGC11859", 9h, HI observations, JVLA C-array\\
2020, VLA/20A-493, "Clues to the Origin of Cold Gas in the Void Elliptical Galaxy KIG 16", 12h (DDT),HI observations, JVLA C-array\\
\textbf{Publications relevant for the proposal:}\\
{\scriptsize{$\bullet$}} Lopez-Rodriguez, E., Mao, S.A., Beck, R., et al. (inc. \textbf{Marcum, P.M.}), 2022, "Extragalactic magnetism with SOFIA (SALSA Legacy Program) IV: Program overview and first results on the polarization fraction" (submitted to ApJ)\\
{\scriptsize{$\bullet$}} Lopez-Rodriguez, E., Clarke, M., Shenoy, S., et al. (inc. \textbf{Marcum, P.M.}), 2022, "Extragalactic magnetism with SOFIA (SALSA Legacy Program) III: FIrst data release and on-the-fly polarization mapping characterization", (submitted to ApJ)\\
{\scriptsize{$\bullet$}} Borlaff, A.S., G{\'o}mez-Alvarez, P., Altieri, B., \textbf{Marcum, P.M.}, et al, 2022, "Euclid preparation XVI: Exploring the ultra-low surface brightness Unvierse with Euclid/VIS", A\&A, 657, 92\\
{\scriptsize{$\bullet$}} Ashley, T., \textbf{Marcum, P.M.}, Alpaslan, M., Fanelli, M.N., Frost, J.D., 2019, "The Neutral Gas Properties of Extremely Isolated Early-type Galaxies III", AJ, 157, 158\\
{\scriptsize{$\bullet$}} Alpaslan, M., Grootes, M.W., \textbf{Marcum, P.M.}, et al., 2016, "Galaxy And Mass Assembly (GAMA): Stellar mass growth of spiral galaxies in the cosmic web", MNRAS, 457, 2287\\
\medskip \hrule \vspace{5pt} \medskip
%================ E N D   I N S T R U C T I O N S ===============
\textbf{\color{Blue}\large Dr. Daniel Dent{\'o}n (Principal Investigator):}\\
University of CBA\\
\vspace{4ex}
\textbf{Education}\\
01012004 - 12312008, University of ABC, Ph.D., astrophysics\\
01012000 - 12312003, University of XYZ, B.S., astronomy\\
\textbf{Appointments}\\
01012012 - present, professor, University of CBA\\
01032009 - 01082011, postdoc, University of ZYX\\
\textbf{Additional Awards, Positions, Fellowships and Proposals}\\
01102011 - 01102012, Awarded 100 hrs on Gigantic Telescope\\
01052011, Best Researcher of the Year\\
\textbf{Publications relevant for the proposal:}\\
{\scriptsize{$\bullet$}} Young, E.T., Herter, T.L., Gusten, R., et al. (inc. \textbf{Denton, D.}), 2021, "Early science results from SOFIA", 8444, 844410\\
{\scriptsize{$\bullet$}} Temi, P., \textbf{Denton, D.}, Miller, W.E., et al., 2018, "SOFIA observatory performance and characterization", SPIE, 84444, 8444414\\
{\scriptsize{$\bullet$}} Gehrz, R.D., Becklin, E.E., et al. (inc.\textbf{Denton, D.}), 2015, "Status of the Stratospheric Observatory for Infrared Astronomy (SOFIA)", Adv. in Space Research, 48, 1004\\
{\scriptsize{$\bullet$}} \textbf{Denton, D.}, Appleton, P.N., Higdon, J.L, 2010, "Large Infrared and Optical Color Gradients in the Carwheel Ring Galaxy: Evidence fo rthe First Epoch of Star Formation in the Wake of an Expanding Ring", ApJ, 399, 57\\
\medskip \hrule \vspace{5pt} \medskip
\textbf{\color{Blue}\large Dr. Sally Smith (Science PI):}\\
University of CBA\\
\vspace{4ex}
\textbf{Education}\\
01012014 - 12312020, University of lmn, Ph.D., astronomy\\
01012010 - 12312014, University of abc, B.S., physics\\
\textbf{Appointments}\\
01012023 - present, research fellow, University of CBA\\
01032020 - 01082022, postdoc, University of ZYX\\
\textbf{Additional Awards, Positions, Fellowships and Proposals}\\
01102011 - 01102012, Awarded 100 hrs on Gigantic Telescope\\
43332, Teaching assistant award\\
\textbf{Publications relevant for the proposal:}\\
{\scriptsize{$\bullet$}} Lopez-Rodriguez, E., Mao, S.A., Beck, R., et al. (inc. \textbf{Smith, S.}), 2022, "Extragalactic magnetism with SOFIA (SALSA Legacy Program) IV: Program overview and first results on the polarization fraction" (submitted to ApJ)\\
{\scriptsize{$\bullet$}} Young, E.T., Herter, T.L., Gusten, R., et al. (inc. \textbf{Smith, S.}), 2021, "Early science results from SOFIA", 8444, 844410\\
{\scriptsize{$\bullet$}} Gehrz, R.D., Becklin, E.E., et al. (inc.\textbf{Smith, S.}), 2015, "Status of the Stratospheric Observatory for Infrared Astronomy (SOFIA)", Adv. in Space Research, 48, 1004\\
{\scriptsize{$\bullet$}} \textbf{Smith, S.}, Appleton, P.N., Higdon, J.L, 2010, "Large Infrared and Optical Color Gradients in the Carwheel Ring Galaxy: Evidence fo rthe First Epoch of Star Formation in the Wake of an Expanding Ring", ApJ, 399, 57\\
\medskip \hrule \vspace{5pt} \medskip
\textbf{\color{Blue}\large Dr. Pamela M. Marcum (co-Investigator):}\\
NASA ARC\\
\vspace{4ex}
\textbf{Education}\\
1994, University of Wisconsin, Ph.D., Astrophysics\\
1989, Florida Institute of Technology, M.S., Space Science\\
1989, Florida Institute of Technology, M.S., Physics\\
1987, Florida Institute of Technology, B.S., Space Science\\
\textbf{Appointments}\\
2016 - present, research scientist, NASA Ames Research Center\\
2020 - 2021, program officer (on detail), NASA HQ, Washington, D.C.\\
2009 - 2016, SOFIA project scientist, NASA Ames Research Center\\
2005 - 2008, program officer (IPA), NASA HQ, Washington, D.C.\\
2003 - 2009, associate professor, Texas Christian University (Fort Worth, TX)\\
1996 - 2002, assistant professor, Texas Christian University (Fort Worth, TX)\\
1994 - 1996, postdoctoral fellow, University of Virginia\\
\textbf{Additional Awards, Positions, Fellowships and Proposals}\\
2018 - present, Roman Space Telescope Standing Review Board\\
2021 - 2022, SOFIA Cycle 9 observing proposal 09\_0113: "Testing for multi-component magnetic fields and their effects on the structure of galaxtic disks", 20.1h, SOFIA/HAWC+ (pending observations/funding)\\
2021, VLA/21A-043,"Flares, breaks and warps in the outskirts of the HI and stellar disk of UGC11859", 9h, HI observations, JVLA C-array\\
2020, VLA/20A-493, "Clues to the Origin of Cold Gas in the Void Elliptical Galaxy KIG 16", 12h (DDT),HI observations, JVLA C-array\\
\textbf{Publications relevant for the proposal:}\\
{\scriptsize{$\bullet$}} Lopez-Rodriguez, E., Mao, S.A., Beck, R., et al. (inc. \textbf{Marcum, P.M.}), 2022, "Extragalactic magnetism with SOFIA (SALSA Legacy Program) IV: Program overview and first results on the polarization fraction" (submitted to ApJ)\\
{\scriptsize{$\bullet$}} Lopez-Rodriguez, E., Clarke, M., Shenoy, S., et al. (inc. \textbf{Marcum, P.M.}), 2022, "Extragalactic magnetism with SOFIA (SALSA Legacy Program) III: FIrst data release and on-the-fly polarization mapping characterization", (submitted to ApJ)\\
{\scriptsize{$\bullet$}} Borlaff, A.S., G{\'o}mez-Alvarez, P., Altieri, B., \textbf{Marcum, P.M.}, et al, 2022, "Euclid preparation XVI: Exploring the ultra-low surface brightness Unvierse with Euclid/VIS", A\&A, 657, 92\\
{\scriptsize{$\bullet$}} Ashley, T., \textbf{Marcum, P.M.}, Alpaslan, M., Fanelli, M.N., Frost, J.D., 2019, "The Neutral Gas Properties of Extremely Isolated Early-type Galaxies III", AJ, 157, 158\\
{\scriptsize{$\bullet$}} Alpaslan, M., Grootes, M.W., \textbf{Marcum, P.M.}, et al., 2016, "Galaxy And Mass Assembly (GAMA): Stellar mass growth of spiral galaxies in the cosmic web", MNRAS, 457, 2287\\
\medskip \hrule \vspace{5pt} \medskip
%================ E N D   I N S T R U C T I O N S ===============
\textbf{\color{Blue}\large Dr. Daniel Dent{\'o}n (Principal Investigator):}\\
University of CBA\\
\vspace{4ex}
\textbf{Education}\\
01012004 - 12312008, University of ABC, Ph.D., astrophysics\\
01012000 - 12312003, University of XYZ, B.S., astronomy\\
\textbf{Appointments}\\
01012012 - present, professor, University of CBA\\
01032009 - 01082011, postdoc, University of ZYX\\
\textbf{Additional Awards, Positions, Fellowships and Proposals}\\
01102011 - 01102012, Awarded 100 hrs on Gigantic Telescope\\
01052011, Best Researcher of the Year\\
\textbf{Publications relevant for the proposal:}\\
{\scriptsize{$\bullet$}} Young, E.T., Herter, T.L., Gusten, R., et al. (inc. \textbf{Denton, D.}), 2021, "Early science results from SOFIA", 8444, 844410\\
{\scriptsize{$\bullet$}} Temi, P., \textbf{Denton, D.}, Miller, W.E., et al., 2018, "SOFIA observatory performance and characterization", SPIE, 84444, 8444414\\
{\scriptsize{$\bullet$}} Gehrz, R.D., Becklin, E.E., et al. (inc.\textbf{Denton, D.}), 2015, "Status of the Stratospheric Observatory for Infrared Astronomy (SOFIA)", Adv. in Space Research, 48, 1004\\
{\scriptsize{$\bullet$}} \textbf{Denton, D.}, Appleton, P.N., Higdon, J.L, 2010, "Large Infrared and Optical Color Gradients in the Carwheel Ring Galaxy: Evidence fo rthe First Epoch of Star Formation in the Wake of an Expanding Ring", ApJ, 399, 57\\
\medskip \hrule \vspace{5pt} \medskip
\textbf{\color{Blue}\large Dr. Sally Smith (Science PI):}\\
University of CBA\\
\vspace{4ex}
\textbf{Education}\\
01012014 - 12312020, University of lmn, Ph.D., astronomy\\
01012010 - 12312014, University of abc, B.S., physics\\
\textbf{Appointments}\\
01012023 - present, research fellow, University of CBA\\
01032020 - 01082022, postdoc, University of ZYX\\
\textbf{Additional Awards, Positions, Fellowships and Proposals}\\
01102011 - 01102012, Awarded 100 hrs on Gigantic Telescope\\
43332, Teaching assistant award\\
\textbf{Publications relevant for the proposal:}\\
{\scriptsize{$\bullet$}} Lopez-Rodriguez, E., Mao, S.A., Beck, R., et al. (inc. \textbf{Smith, S.}), 2022, "Extragalactic magnetism with SOFIA (SALSA Legacy Program) IV: Program overview and first results on the polarization fraction" (submitted to ApJ)\\
{\scriptsize{$\bullet$}} Young, E.T., Herter, T.L., Gusten, R., et al. (inc. \textbf{Smith, S.}), 2021, "Early science results from SOFIA", 8444, 844410\\
{\scriptsize{$\bullet$}} Gehrz, R.D., Becklin, E.E., et al. (inc.\textbf{Smith, S.}), 2015, "Status of the Stratospheric Observatory for Infrared Astronomy (SOFIA)", Adv. in Space Research, 48, 1004\\
{\scriptsize{$\bullet$}} \textbf{Smith, S.}, Appleton, P.N., Higdon, J.L, 2010, "Large Infrared and Optical Color Gradients in the Carwheel Ring Galaxy: Evidence fo rthe First Epoch of Star Formation in the Wake of an Expanding Ring", ApJ, 399, 57\\
\medskip \hrule \vspace{5pt} \medskip
\textbf{\color{Blue}\large Dr. Pamela M. Marcum (co-Investigator):}\\
NASA ARC\\
\vspace{4ex}
\textbf{Education}\\
1994, University of Wisconsin, Ph.D., Astrophysics\\
1989, Florida Institute of Technology, M.S., Space Science\\
1989, Florida Institute of Technology, M.S., Physics\\
1987, Florida Institute of Technology, B.S., Space Science\\
\textbf{Appointments}\\
2016 - present, research scientist, NASA Ames Research Center\\
2020 - 2021, program officer (on detail), NASA HQ, Washington, D.C.\\
2009 - 2016, SOFIA project scientist, NASA Ames Research Center\\
2005 - 2008, program officer (IPA), NASA HQ, Washington, D.C.\\
2003 - 2009, associate professor, Texas Christian University (Fort Worth, TX)\\
1996 - 2002, assistant professor, Texas Christian University (Fort Worth, TX)\\
1994 - 1996, postdoctoral fellow, University of Virginia\\
\textbf{Additional Awards, Positions, Fellowships and Proposals}\\
2018 - present, Roman Space Telescope Standing Review Board\\
2021 - 2022, SOFIA Cycle 9 observing proposal 09\_0113: "Testing for multi-component magnetic fields and their effects on the structure of galaxtic disks", 20.1h, SOFIA/HAWC+ (pending observations/funding)\\
2021, VLA/21A-043,"Flares, breaks and warps in the outskirts of the HI and stellar disk of UGC11859", 9h, HI observations, JVLA C-array\\
2020, VLA/20A-493, "Clues to the Origin of Cold Gas in the Void Elliptical Galaxy KIG 16", 12h (DDT),HI observations, JVLA C-array\\
\textbf{Publications relevant for the proposal:}\\
{\scriptsize{$\bullet$}} Lopez-Rodriguez, E., Mao, S.A., Beck, R., et al. (inc. \textbf{Marcum, P.M.}), 2022, "Extragalactic magnetism with SOFIA (SALSA Legacy Program) IV: Program overview and first results on the polarization fraction" (submitted to ApJ)\\
{\scriptsize{$\bullet$}} Lopez-Rodriguez, E., Clarke, M., Shenoy, S., et al. (inc. \textbf{Marcum, P.M.}), 2022, "Extragalactic magnetism with SOFIA (SALSA Legacy Program) III: FIrst data release and on-the-fly polarization mapping characterization", (submitted to ApJ)\\
{\scriptsize{$\bullet$}} Borlaff, A.S., G{\'o}mez-Alvarez, P., Altieri, B., \textbf{Marcum, P.M.}, et al, 2022, "Euclid preparation XVI: Exploring the ultra-low surface brightness Unvierse with Euclid/VIS", A\&A, 657, 92\\
{\scriptsize{$\bullet$}} Ashley, T., \textbf{Marcum, P.M.}, Alpaslan, M., Fanelli, M.N., Frost, J.D., 2019, "The Neutral Gas Properties of Extremely Isolated Early-type Galaxies III", AJ, 157, 158\\
{\scriptsize{$\bullet$}} Alpaslan, M., Grootes, M.W., \textbf{Marcum, P.M.}, et al., 2016, "Galaxy And Mass Assembly (GAMA): Stellar mass growth of spiral galaxies in the cosmic web", MNRAS, 457, 2287\\
\medskip \hrule \vspace{5pt} \medskip
%================ E N D   I N S T R U C T I O N S ===============

\newpage
%======================================================
%   H O W   T O   U S E   T H I S   F I L E
% (1) download or connect to https://github.com/pmarcum/LaTex-Formatting and select the
%        set of files (aas-style or general) that meet your situation, you can either download the files or link
%        Overleaf to the needed files in the GitHub repository through URL, and then add the following to
%        the preamble of your main paper
%\usepackage{FUpackages}
%\usepackage{FUformatting}
%\usepackage{FUnewCommands} % not needed by WorPT files but generally useful
% (2) type the following into the main latex file where you want to put this table:
\begin{longtable}{|l|p{5in}|}
   \expinput{do_NOT_manually_edit/NOTANONcurrentpending}
\end{longtable}
%================ E N D   I N S T R U C T I O N S ===============


\end{document}

